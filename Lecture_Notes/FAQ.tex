\section{\capitals{[Promela]} What are the exact semantics of ``\texttt{atomic}'' and ``\texttt{d\_step}''?}
The keyword ``\texttt{atomic}'' describes a \emph{weakly}, the keyword ``\texttt{d\_step}'' a \emph{strongly} atomic sequence. The difference lies in the interruption condition: The first can \emph{only} be interrupted if a statement is not executable while the second cannot be interrupted \emph{at all} (see slide 21 in ``Concurrent Programming'').

\section{[General] What is the difference between ``deadlock'', ``livelock'' and ``starvation''?}
Processes are in a \emph{deadlock}, if each process waits for an event that only other processes can trigger. A \emph{livelock} is a concrete deadlock where two or more processes are not waiting, but are trapped in a loop and cannot complete their task. \emph{Starvation} describes the state of a process that is waiting for an event that does not occur.

\section{\capitals{[Promela]} How do the statement types relate to their executability?}
Table \ref{T:Exec} illustrates the answer (see slide 28 in ``Distributed Programming'').
\begin{table}[h]
\centering
\caption{Executability of Statements}
\label{T:Exec}

\begin{tabular}{|l|l|}
\hline
\textbf{Statement Type} & \textbf{Executability} \\
\hline
\emph{assignments} & always \\
\emph{assertions} & always \\
\emph{print statements} & always \\
\emph{expression statements} & iff value is $\lnot 0 \lor \lnot \texttt{false}$ \\
\texttt{send !\ msg} & iff message queue is not full, i.e.\ $n < cap$ \\
\texttt{request ?\ msg} & iff \texttt{request} is not empty, i.e.\ $n > 0$\\
\hline
\end{tabular}
\end{table}

\section{[LTL] What is the meaning of ``$\Box\Diamond\phi$''? How is this justified?}
(see slide 25 in ``LTL (1)'')

\section{[LTL] How are ``$\Box$'' and ``$\Diamond$'' related? What is the intuition?}

\section{[JML] How can \texttt{($\backslash$forall $\tau$ x; a; b)} and \texttt{($\backslash$exists $\tau$ x; a; b)} be translated?}
According to slide 34 in ``JML (1)'', the first statement is equivalent to an \emph{implication}, \texttt{($\backslash$forall $\tau$ x; a {\bf==>} b)}, while the second expression has the same semantics as a \emph{conjunction}, \texttt{($\backslash$exists $\tau$ x; a {\bf\&\&} b)}

\section{[JML] Is ``\texttt{/$*$@ non\_null @$*$/}'' always the opposite of ``\texttt{/$*$@ nullable @$*$/}''?}
No! If an array is specified as \texttt{/$*$@ nullable @$*$/}, then the array itself may be a null pointer or the \emph{elements} of this array can be null. On the other hand, if an array is declared as \texttt{/$*$@ non\_null @$*$/}, then neither the array nor its elements are null (see slide 30 in ``JML (2)'').

\section{[JML] Is ``\texttt{/$*$@ pure @$*$/}'' and ``\texttt{/$*$@ assignable $\backslash$nothing @$*$/}'' the same?}
No! The scope of \texttt{/$*$@ assignable $\backslash$nothing @$*$/} is \emph{local} to each specification case of a method. Specifying a method as \texttt{/$*$@ pure @$*$/} is a \emph{global} statement and also prohibits non-terminationand exceptions (see slide 8 in ``Proof Obligations'').

\section{[DL-Calculus] Why there are three steps in the ``\textsf{loopInvariant}''-rule?}
First, the case \emph{initially valid} proves that the loop invariant holds prior to the loop execution. Second, the step \emph{preserved} ... . Third, the \emph{use case} ... (see slides 7 ff.\ in ``Loop Invariants'').