\section{\capitals{[Promela]} What are the exact semantics of ``\texttt{atomic}'' and ``\texttt{d\_step}''?}
The keyword ``\texttt{atomic}'' describes a \emph{weakly}, the keyword ``\texttt{d\_step}'' a \emph{strongly} atomic sequence. The difference lies in the interruption condition: The first can \emph{only} be interrupted if a statement is not executable while the second cannot be interrupted \emph{at all} (see Slide 21 in ``Concurrent Programming'').

\section{What is the difference between ``starvation'', ``livelock'' and ``deadlock''?}
Processes are in a \emph{deadlock}, if each process waits for an event that only other processes can trigger. A \emph{livelock} is a concrete deadlock where two or more processes are not waiting, but are trapped in a loop and cannot complete their task. \emph{Starvation} describes the state of a process that is waiting for an event that does not occur.

\section{\capitals{[Promela]} How do the statement types relate to their executability?}
Table \ref{T:Exec} illustrates the answer (see Slide 28 in ``Distributed Programming'').
\begin{table}[h]
\centering
\caption{Executability of Statements}
\label{T:Exec}

\begin{tabular}{|l|l|}
\hline
\textbf{Statement Type} & \textbf{Executability} \\
\hline
\emph{assignments} & always \\
\emph{assertions} & always \\
\emph{print statements} & always \\
\emph{expression statements} & iff value is $\lnot 0 \lor \lnot \texttt{false}$ \\
\texttt{send !\ msg} & iff message queue is not full, i.e.\ $n < cap$ \\
\texttt{request ?\ msg} & iff \texttt{request} is not empty, i.e.\ $n > 0$\\
\hline
\end{tabular}
\end{table}

\section{[LTL] What is the meaning of ``$\Box\Diamond\phi$''? How is this meaning justified?}
(see Slide 25 in ``LTL (1)'')