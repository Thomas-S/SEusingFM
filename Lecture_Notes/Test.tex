Für $b \in \{\alpha Q \backslash \alpha P\}$ und $c \in \{\alpha P \cap \alpha Q\}$ gilt die Gleichung:\\
$(c \to P) \ || \ (b \to Q) = b \to ((c \to P) \ || \ Q)$\\

Intuitiv ist $c$ ein Ereignis, dass nur durch gleichzeitiges Ausführen der Prozesse $P$ und $Q$ möglich ist, während das Ereignis $b$ nur im Prozess $Q$ ablaufen kann.\\

Die Gleichung stimmt deshalb, weil $b$ erst den Prozess $Q$ anstoßen muss damit $c$, welches das gleichzeitige Ausführen von $P$ und $Q$ erfordert, ausgeführt werden kann. Mittels Umformung sollte dies zu zeigen sein. Du solltest auch nochmal Deine Herleitung der Gleichheit der Alphabete überprüfen (angesichts der Bedingungen über die Variablen).\\

Die Äquivalenz der Spuren kann wie folgt durch Umformung gezeigt werden:\\
nach Definition von $||$:\\
$\texttt{traces}(P \ || \ Q) = $\\$\{ t \in (\alpha P \cup \alpha Q)^* \ | \ (t \uparrow (\alpha P)) \in \texttt{traces}(P) \ \land \ (t \uparrow (\alpha Q)) \in \texttt{traces}(Q) \}$\\
nach Definition von $\to$:\\
$\texttt{traces}(x \to P) = \{ () \} \cup \{ (x).t \ | \ t \in \texttt{traces}(P)\}$\\
linke Seite (1. Glied):\\
$\texttt{traces}(c \to P) = \{ () \} \cup \{ (c).t \ | \ t \in \texttt{traces}(P)\}$\\
linke Seite (2. Glied):\\
$\texttt{traces}(b \to Q) = \{ () \} \cup \{ (b).t \ | \ t \in \texttt{traces}(Q)\}$\\
linke Seite:\\
$\texttt{traces}((c \to P) \ || \ (b \to Q)) = $\\
rechte Seite:\\
$\texttt{traces}(b \to ((c \to P) \ || \ Q) = $\\